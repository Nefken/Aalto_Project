\documentclass[a4paper,11pt]{article}
\usepackage[T1]{fontenc}
\usepackage[utf8]{inputenc}
\usepackage{lmodern}
\usepackage[english]{babel}
\usepackage[top=2.5cm,bottom=2.5cm,left=2.5cm,right=2.5cm]{geometry}
\begin{document}

\begin{center}
    {\Large \textbf{Presentation story}}
\end{center}

{\Large \textbf{Slide 1 : Titre}}

Hello everyone, welcome to this presentation. The goal today is to describe what was learned and made during this application internship which was realized within the PICO Group at Aalto University under Pr. Pekola's supervision who is  among us today. The name of the project is Fabrication and measurements of NIS Junctions in order to characterize Plasma Etching.
\medbreak

{\Large \textbf{Slide 2 : Outline}}

First of all, here is the outline of this presentation. We will start with an introduction about the group, and the context of this internship. We will really start with a bit of theory to remind the situation. We will pursue with experimental methods that were used to make the structures we measured. Then, it is important to talk more specifically about the experimental protocol we made. Finally, we will present the several results that came out.
\medbreak

{\Large \textbf{Slide 3 : Introduction}}

The PICO group is part of the Aalto University in Espoo, Finland. The main topics that are studied are thermometry and heat transport at low temperature. One of the goals of the group is to correctly define the ampere, the units to measure current. In the cleanroom reseasrchers of the group us several tools to make structures. One of them is the evaporator, nicknamed LISA, here in the picture. It is a recent acquisition in the cleanroom, so all its features are not yet characterized. Particularly the plasma gun which allows plasma etching. This was the goal of this internship : characterize plasma etching.
\medbreak 

{\Large \textbf{Slide 4 : Theory}}

The junctions realized rely on superconductivity. The most common things we know about superconductivity is two properties : zero electric resistance and perfect diamagnetism. These properties can be explained thanks to the BCS theory of superconductivity stated by Bardeen Cooper and Schrieffer. This theory says that electron in a ion lattice can move, when they move, they attract the ion lattice creating a positively charged area. This charge attracts another electron from far away in the matter. The two electrons are then linked, forming a cooper pair, which is no longer a fermion but a boson, following the bose-einstein distribution. This has for effect to open a gap in the density of stats of the electrons within the superconductor.
\medbreak

{\Large \textbf{Slide 5 : Theory}}

When we put materials with different density of states in contact, they equalize their Fermi level, like of this figure. The hatched area are the filled states. In this case, there can't be any tunneling, but, when we apply a bias voltage, which will move up or down these density of states. Then electrons will be able to tunnel, thus inducing a current trough the junction. Yet, this density of states does not take into account some defaults, like rough surface between materials, alignement mismatch... Dynes introduced another density of states which take these defaults inot account and where the gap contains some available states, permitting tunneling for the electrons even for a subgap voltage. The induced current is called leakage current. The ratio of this current and the tunneling current gives an idea of the quality of the junction.
\medbreak

{\Large \textbf{Slide 6 : Experimental methods}}

Now, we will talk about the tools and methods we used in the cleanroom to make the devices. We want to realize NIS junctions, with Aluminium and Copper. To do so, we need to depose metal in a particular area on the wafer. This can be done with Electron Beam lithography, by exposing resist. Resists are deposed and spinned then baked. They are exposed in the EBL according to a pattern previously designed and during development, exposed resists are withdrawed by chemicals, as you can see a cross section.
Then, the wafer is ready for the deposition which is done in the LISA evaporator. There are holes in the resists, which is important to not evaporate metal on all the wafer but, we still want to make junctions, so we can't just evaporate metal, we need to set an angle, to evaporate below the remaining resist, inside the undercuts, like in the figure. 
\medbreak 

{\Large \textbf{Slide 7 : Experimental methods}}

Some oxygen can be pumped in the chamber to oxidize the Al. And we can use plasma for two applications : at low power to ease the lift-off and at high power to etch samples. Once Copper is evaporated, which is the last step in the evaporator (for our process), we let the sample in aceton for lift-off. Finally, we can use the SEM to check the sample; here some images of samples in both normal and secondary electron mode. And here a failed sample, rather different.
\medbreak 

{\Large \textbf{Slide 8 : Dilution Cryostat}}
Cool down
\medbreak

{\Large \textbf{Slide 9 : Experimental protocol}}

Here is the experimental protocol followed, with the parameter used. 20 samples ar for statistics, then we will make several type of samples.

{\Large \textbf{Slide 10 : Results, reference samples}}

First of all, before trying to measure any fancy sample, we need references made with our particular process. We made a clean contact to measure the resistance of the leads, which is consistent with what the calculus gives, about 70 Ohm. 
Another important reference we need is a strong oxidation, to know if we etch the native oxide or not. Conductance proportionnal to S, 7kOhm

Idem tunnel barrier.
\medbreak

{\Large \textbf{Slide 11 : Results, Position}}

Then, come the first tests with plasma, which goal was to determine if the plasma was homogeneous on all the sample holder. If someone want to etch a whole wafer for exemple. According to the results we had, it seems homogeneous. Yet, there is another conclusion we can get from this curve : no surface dependance, and order of magnitude close to the clean contact : we can assume all the oxide was etched.
\medbreak

{\Large \textbf{Slide 12 : Results, Time}}

\medbreak

{\Large \textbf{Slide 13 : Plasma problems}}

\medbreak

{\Large \textbf{Slide 14 : Wafer Etching}}

\medbreak

{\Large \textbf{Slide 15 : Tunnel before}}

\medbreak

{\Large \textbf{Slide 16 : Tunnel plasma}}

\medbreak

{\Large \textbf{Slide 17 : Tunnel after}}

\medbreak

{\Large \textbf{Slide 18 : Conclusion}}

\medbreak

{\Large \textbf{Slide 19 : Questions}

\end{document}
