\begin{center}
\vspace*{\stretch{2}}
{\Large \textsc{Remerciements}}
\vspace{0.5cm}

\textit{Je tiens tout d'abord à remercier le Professeur Jukka Pekola pour avoir accepté ma demande d'intégrer son équipe. Bien sûr cela n'aurait pas été possible sans l'aide du Docteur Clemens Winckelmann qui m'a fournit ses coordonées. De plus, je souhaite remercier l'ensemble du PICO Group pour m'avoir si bien accueilli tout au long de ce séjour, et plus particulièrement le Docteur Mathieu Taupin, qui m'a intégré au sein de son projet et m'a fourni nombres de conseils en salle blanche. Enfin, je souhaite remercier Grenoble-INP PHELMA qui m'a donné l'opportunité de réaliser ce stage.}
\vspace*{\stretch{9}}
\end{center}
\newpage

\section*{Le PICO Group}

Le PICO Group est une équipe de recherche qui exerce sous la tutelle de l'Université d'Aalto. Elle possède ses locaux au sein de la plateforme \textsc{Micronova} situé sur le campus d'Aalto à Espoo (Finlande) : Otaniemi, qui regroupe l'ensemble des activités scientifiques de l'Université. L'équipe est actuellement composé de dix-neuf personnes, dont le Professeur Jukka \textsc{Pekola}, qui dirige le groupe et le Docteur Matthias \textsc{Meschke} qui est responsable des appareils. L'équipe est ensuite composée de cinq docteurs, sept diplomés actuellement en thèse ainsi que cinq étudiants accueillis pour l'été. 

Les sujets de recherches étudiés par le groupe sont divers mais se basent tous sur les propriétés des supraconducteurs. Les principaux sujets d'études sont liés à la thermodynamique quantique et au transport d'électrons au sein de structures quantiques. Ainsi, certains chercheurs sont focalisés sur une dimension théorique : statistiques d'échanges thermiques entre deux résistances\cite{statistics}, par exemple, tandis que d'autres abordent davantage une partie pratique : réalisation de structures afin d'en déterminer les propriétés, refroidissement d'électrons... De plus, le groupe est en étroite collaboration avec le centre de métrologie car l'un des principaux projet en cours est la redéfinition de l'Ampère\cite{ampere}, unité de mesure du courant électrique. Pour pouvoir couvrir l'ensemble de ces sujets, l'équipe a besoin de certains appareils.

Le groupe est également appelé Low Temperature Laboratory, car une grande partie des expériences réalisées sont faites à basse température. Ainsi, le groupe possède trois cryostats à dilution He$_3$/He$_4$ et un cryostat à dilution sèche (BlueFors). De plus, le bâtiment possède sa propre salle blanche afin de fabriquer les structures à étudier, avec notamment des Electron Beam Lithographier (EBL), Evaporateurs, Scanning Electron Microscope (SEM), Atomic Layer Deposition (ALD)...

L'université et le batiment fournissent ainsi toutes les ressources nécessaires au recherches effectuées au sein du groupe.


\chapter*{Introduction}
\addcontentsline{toc}{chapter}{Introduction}

Nanofil... un terme très à la mode dans le milieu scientifique de la recherche à l'heure actuelle. Un nanofil est une structure de taille nanométrique et généralement de forme cylindrique ou du moins prismatique. Les propriétés qui découlent de cette forme particulière et de cette taille caractéristique trouvent majoritairement leur origine dans la théorie quantique, c'est pour cela que ces structures intéressent tant. Elle permettent de mieux comprendre certains phénomènes quantiques mal compris, et même d'utiliser les propriétés quantiques de la matière ou obtenir des fonctions électroniques, notamment. Le recherche sur les nanofils aura sans doute une application dans le monde pratique dans quelques années voire décennies mais pour cela elle doit davantage progresser. C'est ainsi que nous nous proposons d'étudier des propriétés de supraconduction induite au sein de nanofils semiconducteurs.

Une équipe de Copenhague a mis au point une méthode de croissance de nanofils d'InAs/Al par épitaxie\cite{epitaxie} (conservation des directions cristallographiques). L'InAs est un matériau semiconducteur alors que l'Aluminium est supraconducteur à basse température ($T_c=1.2K$), des phénomènes de supraconductivité de proximité induite vont intervenir au sein du nanofil. C'est ce phénomène que nous souhaiterions observer pour vérifier s'il concorde avec la théorie, qu'il faudra préalablement étudier et comprendre (Chapitre \ref{Chap1}). Ainsi, l'équipe a également réalisé des structures en intégrant ces nanofils, cependant ces structures possèdent des éléments parasites qui ne permettent pas de caractériser correctement les nanofils : des quantum dots apparaissent aux extrémités (Voir Annexe Fig.\ref{StructureCopenhague}). Pour annuler leurs effets, il est possible de polariser ces zones en appliquant une forte tension via une capacité (Voir Annexe Fig.\ref{StructureCopenhagueDéfauts}). Cependant, il est impossible de savoir dans quelle mesure ces capacités se couplent à celles permettant les vraies mesures. Pour une meilleure caractérisation, il faudrait parvenir à isoler les nanofils et à les intégrer à une structure propre. Le but de ce projet est alors de mettre en place un processus de salle blanche visant à intégrer les nanofils réalisés par l'équipe du Dr. Krogstrup puis de les caractériser par pompage d'électrons.

Cependant, il est totalement impossible de se lancer directement dans l'intégration des nanofils sans avoir préalablement établi un processus potentiellement fonctionnel. Pour cela, il nous faudra une bonne connaissance des appareils disponibles en salle blanche, pour distinguer ce qui est réalisable de ce qui ne l'est pas (Chapitre \ref{Chap2}).

De plus, le fait d'avoir un processus viable ne suffit pas, il faut connaître les différents paramètres qui entrent en jeu pour la réalisation de la structure. Pour cela, il est impératif de se livrer à un certain nombre de tests sur des structures plus simples notamment pour valider l'efficacité des certaines étapes en fonction des paramètes appliqués. Ainsi, la réalisation des tests permettra de mieux connaître les appareils de salle blanche, de nous familiariser avec ceux-ci, mais constituera aussi une base solide sur laquelle s'appuyer pour commencer l'intégration des nanofils car elle fournira des informations importantes concernant les paramètres à utiliser dans chaque situation (Chapitre \ref{Chap3}).

Munis de ces outils, nous pouvons nous consacrer au but principal de ce projet à savoir l'intégration, puis la caractérisation des nanofils. Une fois qu'au vu des différents tests réalisés, le processus de fabrication semble opérationnel, il nous est possible de le mettre en place. Cependant, comme les strucures sont différentes, il faut s'attendre à certains ajustements des paramètres pour obtenir ce que nous souhaitons. Ainsi, une fois les structures réalisées, nous pouvons commencer la caractérisation et les mesures (Chapitre \ref{Chap4}).

Pour finir, la recherche ne s'arrête pas ici, il reste encore beaucoup à faire, en fonction des résultats obtenus et de leux validité vis-à-vis de la théorie, il est possible d'aller plus loin et d'ouvrir le sujet à d'autres perspectives. L'étude des phénomènes de Fermions de Majorana au sein des interfaces de ces nanofils en est un par exemple. Le recherche n'est pas figée, et il est toujours possible d'aller plus loin et d'ouvrir la porte à des applications pratiques et/ou à de la recherche plus fondamentale (Chapitre \ref{Chap5}).


% Tout d'abord, en arrivant, les nanofils auront subit un voyage et seront oxydés. Il faut ainsi un moyen de retirer l'oxyde sans altérer le nanofil. 

%Ce projet s'intitule Caractérisation de nanofils semiconducteurs (InAs) par pompage d'électrons. Il prend place au sein de la recherche actuelle sur les nanofils, à mi-chemin entre l'élaboration de nanostructures et l'étude théorique des phénomènes physiques qui interviennent, il couvre un large panel de la physique des laboratoires d'aujourd'hui : de la conception de structures à la caractérisation en basse température, en passant par la fabrication en salle blanche, il s'inscrit dans mon projet professionnel en tant que complément pratique par la manipulation d'appareils et le travail en salle blanche mais aussi théorique sur l'étude de matériaux différents de ceux étudiés en cours. Une équipe de Copenhague a mis au point une méthode de fabrication de nanofils d'InAs par épitaxie. Notre but est de caractériser leurs échantillons. Le but de ce projet sera donc de mettre en place un process de salle blanche permettant de caractériser les nanofils : les intégrer à une structure permettant des mesures. Pour cela, il faut comprendre les différents dispositifs de salle blanche. De plus, avant de pouvoir complètement caractériser les nanofils, il faudra réaliser de nombreux tests sur des échantillons différents afin de mettre en place le process flow à ce dont nous avons besoin puis des tests sur des nanofils pour l'ajuster.
