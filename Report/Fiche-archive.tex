\documentclass[a4paper,11pt]{article}
\usepackage[T1]{fontenc}
\usepackage[utf8]{inputenc}
\usepackage{lmodern}
\usepackage[francais]{babel}
\usepackage{graphicx}

\begin{document}

\vspace*{\stretch{2}}

    \begin{center}
    
    \hspace*{\stretch{0.2}}
        \includegraphics[width=45mm]{logo_phelma.png}
        \hfill
        \includegraphics[width=35mm]{logo_aalto.png}
    \hspace*{\stretch{0.2}}
        \vspace{2cm}
        
        {\LARGE Nicolas \textsc{Paillet}}\\
        \vspace{0.5cm}
        
        \textsc{Filière} Physique \& NanoSciences (PNS)\\
        Année scolaire 2014-2015\\
        \vspace{1.5cm}
        
        {\LARGE \textsc{Rapport de Stage d'Application}}
        \vspace{0.4cm}
        
        {\Huge \textsc{Fabrication et Mesures de}\\        
        \textsc{jonctions Métal-Isolant-Supraconducteur}
        \vspace{0.3cm}
        
        \textsc{dans le but de caractériser}
                      
        \textsc{une méthode de gravure par Plasma}}
        
        \vspace{2cm}
        
        Stage réalisé du 18 Mai 2015 au 18 Août 2015\\
        
    \vspace{0.3cm}
    
    Au sein du PICO \textsc{Group}
    \vspace{0.5cm}
    
    \includegraphics[width=25mm]{logopico.png}
    \vspace{1cm}
    
    \flushleft{
    \textbf{Sous la supervision de}\\
    \vspace{0.3cm}
    
    Pr. Jukka \textsc{Pekola}, en tant que tuteur de stage\\
    \textit{jukka.pekola@aalto.fi}\\
    \& \\
    Quentin \textsc{Rafhay}, en tant que tuteur école\\
    \textit{quentin.rafhay@phelma.grenoble-inp.fr}\\
    \vspace{1cm}
    }
       \end{center} 
    \vspace*{\stretch{3}}

    \newpage
    
    \begin{Large}
    \textbf{Réalisation et mesures de jonctions Métal - Isolant - Supraconducteur pour la caractérisation de la gravure par plasma}
\end{Large}

\vspace{0.3cm}

Le but de ce stage est de caractériser une méthode de gravure permettant de retirer \textit{in situ} une couche de matière non souhaitée : la gravure par plasma. Ainsi, un processus de fabrication a été mis en place afin de réaliser des structures (jonctions Métal-Isolant-Supraconducteur) à mesurer, incluant, ou non (échantillons référence), la gravure par plasma. il s'appuie sur différentes techniques de salle blanche, en particulier l'évaporateur permettant de déposer le metal mais aussi de graver avec le plasma. Des mesures de résistances et des tracés de caractéristiques courant-tension ont été réalisées, à la fois à temperature ambiante et à basse température, atteinte par la biais d'un cryostat à dilution. Les échantillons de référence qui consistent en de simples jonctions sans gravure montrent un comportement conforme à celui attendu par la théorie : dépendance en surface de la résistance, courant tunnel et courant de fuite à basse température... Le nettoyage du pistolet à plasma par oxygène a perturbé les résultats concernant la gravure par plasma. Cependant, on notera que le plasma est capable de graver l'Oxyde d'Aluminium, qu'il est isotrope sur l'ensemble du porte-échantillon, et que les jonctions, ne sont visiblement pas endommagées par la gravure. Ce stage a donc permis la caractérisation de plusieurs paramètres concernant la gravure par plasma au sein de l'evaporateur et aide à la recherche grâce à cette technique désormais caractérisée.

    \end{document}
