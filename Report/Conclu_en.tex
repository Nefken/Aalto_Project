\pagestyle{fancy}
\chapter*{Conclusion}
\addcontentsline{toc}{chapter}{Conclusion}

In research, it is important to be able to rely on the tools used to fabricate or measure. Manufacturers provide data to give general informations about the tools, yet parameters do not always give the beahavior of the tools. This is why characterization of tools is a crucial part of research to have a solid base to start. 

The goal of the project was to determine if plasma etching is a reliable method to etch matter \textit{in situ}, since it can be useful for some fabrication processes and so characterize the behavior of the plasma in the evaporator nicknamed LISA, a recent acquisition of the cleanroom.

We first learned about the theoretical background on which the phenomenon rely, to be able to understand what happens. This was possible through a bibliography study on the fields involved. Then, an important part of the project have been devocated to learning and then being able to use the necessary tools in the cleanroom, either the functionning of the tools as the practical use of each one. In the meanwhile, the process was set up, all the settings determined and kept constant to have a reliable and systematic method, punctuated by maintenance interruptions. So, we could make measurements on the structures realized both at room and low temperature because the two of them give different informations about etching. According to the results the measurements gave, plasma etching is a reliable method to etch matter \textit{in situ}. We were able to determine its isotropy on the sample stage and assume that it does not damages the junctions. Yet, we were not able to determine a perfect etching duration, since it seems heavily dependant on the cleanliness of the plasma gun. 

However, the results obtained still give important information about plasma etching and can help for future processes that could involve it.

% According to the results I had, I have many information about the Plasma Gun of the LISA evaporator, and I can tell its story of the three months. First, it worked quite properly, ten minuts of plasma where good to etch a native aluminium oxide layer, and all the samples had a good leakage. Then the plasma started to have problems, it burned the resist several times, phenomenon that other members of the group also had. Yet, there was no problems when we did not use the Plasma. It was decided to clean it with Oxygen. The cleaning made it more powerful as less than 5 minutes of Plasma were good to etch the native aluminium oxide layer and ten minutes etched all the aluminium and the wafer. The samples without Plasma still have a good leakage. Then, all samples, even those without Plasma started to have a poor leakage. However, according to the results I had, the Plasma does not damage the quality of the junction since the leakage were about the same order of magnitude for Plasma etched samples and regular ones. Since everyone in the group had troubles with the LISA evaporator, it was decided to stop it, clean it and restart everything to try to solve the problem. It was just before the end of my internship so that I could not participate to the "check" measurements, to see if everything was fine.
%
% This basically conclude on the measurements and results. Apart from this, this internship was an opportunity for me to discover the world of research in Universities. Moreover, I got trainings and became autonomous on several clean room devices. I know a lot more about clean room work and procedures, so that it really comes in addition to the practical formation the school provided me. Then, I had to totally learn from nothing about superconductors since this field os not part of the courses in second year at least. This was interesting to learn from people of the lab because it is quite different from lessons at schoo, less formal. In addition of these conversation I could read articles to understand more some points, or go deeper about one point. 
