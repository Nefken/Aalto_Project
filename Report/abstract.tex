\pagestyle{empty}
\newpage
\section*{Abstract}

\begin{Large}
\textbf{Fabrication and measurements of Normal Metal - Insulator - Superconductor junctions to characterize plasma Etching}
\end{Large}

\vspace{0.3cm}

The goal of this internship is to characterize an etching method that can be performed \textit{in situ} to get rid of an unwanted layer of matter : Plasma Etching. Thus, a fabrication process was put in place in cleanroom in order to make the structures (Normal Metal - Insulator - Superconductor junctions) to measure. This process relies on several cleanroom methods, but particularly on the evaporator that allows metal deposition and plasma etching. The measurements done are mostly resistance and current-voltage measurements both at room and low temperature, reached with a dilution cryostat. The reference samples consists in simple NIS junctions, without plasma etching shows results expected by theory : surface dependance of resistance, tunnel and leakage current... Results and conclusions about plasma etching were bothered by some parameters modification (Oxygen cleaning of the plasma gun). Yet, it is interesting to notice that plasma etching is a valide etching method since it can etch Aluminium Oxide, which is isotropic on all the sample stage surface and which is does not imply any damages in the junctions. This internship could determine several key data about plasma etching in the evaporator and help researcher with a new and characterized method to etch.

Key words : Plasma Etching, Evaporator, NIS junctions, Low temperature measurements.

\section*{Résumé}

\begin{Large}
    \textbf{Réalisation et mesure de jonctions Métal - Isolant - Supraconducteur pour la caractérisation de la gravure par plasma}
\end{Large}

\vspace{0.3cm}

Le but de ce stage est de caractériser une méthode de gravure permettant de retirer \textit{in situ} une couche de matière non souhaitée : la gravure par plasma. Ainsi, un processus de fabrication a été mis en place afin de réaliser des structures (jonctions Métal-Isolant-Supraconducteur) à mesurer, incluant, ou non (échantillons référence), la gravure par plasma. Il s'appuie sur différentes techniques de salle blanche, en particulier l'évaporateur permettant de déposer le metal mais aussi de graver avec le plasma. Des mesures de résistances et des tracés de caractéristiques courant-tension ont été réalisées, à la fois à temperature ambiante et à basse température, atteinte par la biais d'un cryostat à dilution. Les échantillons de référence qui consistent en de simples jonctions sans gravure montrent un comportement conforme à celui attendu par la théorie : dépendance en surface de la résistance, courant tunnel et courant de fuite à basse température... Le nettoyage du pistolet à plasma par oxygène a perturbé les résultats concernant la gravure par plasma. Cependant, on notera que le plasma est capable de graver l'Oxyde d'Aluminium, qu'il est isotrope sur l'ensemble du porte-échantillon, et que les jonctions ne sont visiblement pas endommagées par la gravure. Ce stage a donc permis la caractérisation de plusieurs paramètres concernant la gravure par plasma au sein de l'evaporateur et aide à la recherche grâce à cette technique désormais caractérisée.

Mots clés : Gravure par plasma, Evaporateur, Jonctions NIS, Mesures à basse température.
