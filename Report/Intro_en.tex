\pagestyle{fancy}
\chapter*{The PICO Group}
\addcontentsline{toc}{section}{The PICO Group}

\indent The PICO Group is a research team which works under the responsibility of the Aalto University School of Science. Its facility lies in the Micronova building, situated on the Aalto Campus of Otaniemi, in Espoo, Finland. Otaniemi is the centre of all scientific activites of the Aalto University and the Micronova building holds a major part of research about nano and microphysics in Finland. The PICO Group consists of nineteen members for the moment, including, professor Jukka Pekola, who leads the group, and doctor Matthias Meschke, in charge of the laboratory. Then, the group is divided between five post-doc, seven PhD students and five summer students. 

The main topics are quantum thermodynamics and single electron transport through nano and microscaled devices. The topics studied are both on the experimental and theoretical sides. The group focuses for example on the theory of heat transfert between two nanoscale resistors\cite{statistics} or the practical realization of electron counting devices. Moreover, the group works actively with the Centre for Metrology and Accreditation (MIKES) because one of their goal is to redefine the ampere, the unit used to measure electrical current and still missing in a metrological point of view (measurements not accurate enough so far)\cite{ampere}.

The practical realization of the different devices is possible thank to the presence in the building of an cleanroom with a classification from ISO6 to ISO4. A large panel of equipments is available, but for the project of the intership, only two Scanning Electron Microscope (SEM), one for imaging and one equipped with Electron Beam Lithography (EBL), and an evaporator for metal deposition will be needed. The cleanroom also holds devices for making semiconductors-based structures and the Atomic Layer Deposition device was invented here. The group has a low temperature laboratory with several dilution cryostats. Indeed, since the thermal agitation should be smaller than the characteristic energies of the mesoscopic system (superconducting gap, charging energy... typically $\sim$ meV), they need to reach low temperatures\cite{NIS_thermo}. There are three dilution cryostats that can reach a temperature of 50mK, and a BlueFors dry cryostat which works without Helium bath, reaching 10mK. Other characterization equipments are present, such as a probe station, or an Atomic Force Microscope.

The University and these devices provide the Group the possibility to make research efficiently as we can see the numerous publications published every year in famous journals.


\chapter*{Introduction}
\addcontentsline{toc}{chapter}{Introduction}

%Nanostructures are more and more complicated. Researcher in this field always tend to try to make sophisticated devices with many functionalities. This is possible thanks to clean room devices that provide the researchers a way to make more and more different processes. Yet, some of functionnalities in the devices are not well characterized since researchers tend to focus on their actual structures rather than on the characterization of the devices, especially when devices are quite new. However, it is important to have access to all the possibilities that a device offers, this is why it is important to characterize them. 

Research in microstructures is very active since it provides a way to understand the properties of matter at atomic scale which gices access to numerous applications, such as transistors or detectors. Results in this field may offer new possibilities for the future, in research and also in everyday's life. But, in order to make microstructures, a research group needs several devices for which we know the functionning. Indeed, having a device with a large amount of functionnalities is important but knowing how all these functionnalities work is more important. When a device is new in a laboratory, the first step is to characterize each functionnality to be sure of the behaviour of the tool. 

The evaporator, surnamed LISA, is a recent acquisition in Micronova's cleanroom, so that all its functionalities were not characterized, for example it allows plasma etching, which is an \textit{in situ} etching that can be use to get rid of a layer of matter that is not wanted on the structure. There are other etching methods, like chemical etching but these are \textit{ex situ}, meaning that the etched surface can get contaminated during the transport from on instrument to another. For example, Aluminium is a metal that oxidize very fastly (few dozens of second) in contact with air, getting rid of Al oxide with an \textit{ex situ} method is not worth since at the second the sample will be in contact with air, it will oxidize again. 

Before using plasma etching on real samples, and as it is a new technique, it is important to characterize it : is the plasma uniform on all the sample holder, does it affect the quality of the junction ? In order to answer these questions, we will study simple structures : Normal metal-Insulator-Superconductor junction (NIS junction).

The report is divided as follow. First of all, a theory part will sum up the literature to situate the state of the art and remind the main characteristics of the devices (Chapter \ref{Chap1}).

Then, we will focus on the experimental tools that were used to make the structures : the goal of this part is to present the functionning of all the devices useful for the process, to show why each step is important and start to define the parameters that need to be set (Chapter \ref{Chap2}).

%Then, since I will fabricate some structures, I need to know how to do it. I will have trainings to use several devices in the clean room, but I need to understand how they really work, not just knowing on what button to push. Why do we use this device with this particular way to proceed, what are the parameters than have to be taken into account (Chapter \ref{Chap2}). 

With this knowledge of the tools, we can build an experimental procedure to characterize the plasma : several parameters for the fabrication, consistents and relevant measurements, and design of the measurements (Chapter \ref{Chap3}).

%Once I know how to the devices work, I need to set some parameters to make consistent measurements and characterize the other parameters with changing only one at a time : I need to build an experimental procedure. I also need to put in place a set up to make measurements : what type of measurements should be done, and how can we measure it (Chapter \ref{Chap3}).

Finally, the last part will focus on the results of the measurements, and the exploitation of these results that allows to draw conclusions about plasma etching (Chapter \ref{Chap4}).

%Then, the measurements will give me some datas, which I will need to exploit to turn them into results which I can interpret to understand what happens (Chapter \ref{Chap4}).
