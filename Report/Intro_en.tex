\begin{center}
\vspace*{\stretch{2}}
{\Large \textsc{Aknowledgments}}
\vspace{0.5cm}

\textit{First of all, I would like to thank Pr. Jukka Pekola for having accepted me within the PICO Group and having given me the possibility to do this internship. Of course this would not have been possible without Dr. Clemens Winckelmann who gave me Pr. Pekola contact details. Then, I will of course thank the whole PICO Group for having integrated me so well, but especially Dr. Mathieu Taupin who I worked with on the nanowires project and who gave me precious advices about clean room processes and laboratory methods, alongside with Dr. Matthias Meschke on that last point. Finally, I would like to thank both \textsc{Grenoble INP Phelma} for having given me the opportunity to do such an internship and Aalto University which made it possible afterall.}
\vspace*{\stretch{9}}
\end{center}
\newpage

\section*{The PICO Group}

\indent The PICO Group is a research team which work under the responsibility of the Aalto University School of Science. Its facilities lies in the Micronova building, situated on the Aalto Campus of Otaniemi, in Espoo. Otaniemi is the centre of all scientific activites of the Aalto University and the Micronova building holds the major part of research about nano and microphysics in Finland. The team consists of nineteen members for the moment, including, of course the professor Jukka Pekola, who leads the group and doctor Matthias Meschke who is in charge of the laboratory. Then, the team is divided between five post-doc, seven PhD students and five summer students. 

The Group's researches are varied but especially lies on superconductors properties. The main topics are linked with quantum thermodynamics and single electron transport through nano and microscaled devices. Among the group there are theoretical physicists who focus on the theoretical aspect of research : statistics that rule heat transfert between two nanoscaled resistors\cite{statistics}, for example, whereas some other are focused on practical research : realization of NIS thermometers, electron counting through devices... Moreover, the Group work actively with the Centre for Metrology and Accreditation because one of their main goal is to redefine the ampere, the unit used to measure electrical current\cite{ampere}.

All theses topics of research can be reached thanks to the devices the Group and the building hold. The building owns a clean room, widely used by the members of the Group to make structures. The devices they use are mostly the Electron Beam Lithographier(EBL), evaporators LISA and MASA, and the Scanning Electron Microscope(SEM). The clean room also have devices for making semiconductors-based structures and the Atomic Layer Deposition device was invented here. Then, the Group have a dedicated room in which there are many machines and especially several dilution cryostats. Indeed, since most of the work done within the group has a link with superconductors, they need to reach low temperature to have access to this particular state of matter. There are three regular dilution cryostats that can reach temperature as low as 20mK, and a BlueFors cryostats that works differently but can reach even lower temperatures. In other rooms, there are also devices to characterize and measure : probestation, Atomic Force Microscope (AFM), and SEM, bounders... 

The University and these devices provide the Group the possibility to make research efficiently as we can see the numerous publications published every year in famous journals.


\chapter*{Introduction}
\addcontentsline{toc}{chapter}{Introduction}

\indent Nanowire... Quite a famous word within the research field nowadays. Yet, nanowires are still misunderstood. A nanowire is a nano or microscaled structure and shaped as a cylinder and at least a prisme. The properties that come from this particular shape and size mostly find their origins in the quantum theory and this is the main reason why these devices interests so much. They allow to understand better some aspects of the quantum theory that are still misunderstood and maybe they can lead to use some quantum properties in practical applications. Research about nanowire will eventually find practical applications in years or decades (according to past the observation of semiconductors theory for example), but to achieve that, it needs to make progress. This is why we propose to study the properties of induced superconductivity within a semiconductor (InAs) nanowire.

Peter Krogstrup, a researcher in the University of Copenhaguen put in place a process to growth InAs nanowires epitaxially to Al\cite{epitaxie}. The InAs is a semiconductor whereas Aluminium is superconductor at low temperatures, induced superconductivity will appear within the nanowire who will become superconductor. It is this phenomenon we would like to characterize but to achieve that, we will need to study and understand the theories behind it (Chapter \ref{Chap1}).

The goal we aim is to integrate the nanowire in a Normal metal-Insulator-Superconductor-Insulator-Normal metal (NISIN) junction to study the Andreev bound States that appear in the insulator barrier. The Copenhaguen team made structures with the nanowires but, we saw that there were troubles with the gates. We would like to integrate the structures here, and for this we need what techniques are available in the clean room to make structures (Chapter \ref{Chap2}). 

Once we know what we can do, we need to define some parameters and attribute values to them. This is the role of the several tests we have made : understand the effect of each parameter on the structure. For this, we did not use the nanowires but some very simple structures were we can easily understand what happen if we see any problem (Chapter \ref{Chap3}) Moreover, these tests will teach us to deal with the clean room devices and get used to use them.

Then, we can start to try to integrate the nanowires to a viable structure. According to the tests, the process seems accurate but there will be some adjustments to do, obviously since the conditions are different(Chapter \ref{Chap4}).

Finally, the project does not stop here, there is still a lot to do. First of all, it is important to confront the results with the theory : it is possible to go further and open the project to other perspectives. For example the study of Majorana's fermions within the Semiconductor-Superconductor interface. Research never stops, there is and will always a further point to reach, that what makes it so beautiful (Chapter \ref{Chap5}).


% Tout d'abord, en arrivant, les nanofils auront subit un voyage et seront oxydés. Il faut ainsi un moyen de retirer l'oxyde sans altérer le nanofil. 

%Ce projet s'intitule Caractérisation de nanofils semiconducteurs (InAs) par pompage d'électrons. Il prend place au sein de la recherche actuelle sur les nanofils, à mi-chemin entre l'élaboration de nanostructures et l'étude théorique des phénomènes physiques qui interviennent, il couvre un large panel de la physique des laboratoires d'aujourd'hui : de la conception de structures à la caractérisation en basse température, en passant par la fabrication en salle blanche, il s'inscrit dans mon projet professionnel en tant que complément pratique par la manipulation d'appareils et le travail en salle blanche mais aussi théorique sur l'étude de matériaux différents de ceux étudiés en cours. Une équipe de Copenhague a mis au point une méthode de fabrication de nanofils d'InAs par épitaxie. Notre but est de caractériser leurs échantillons. Le but de ce projet sera donc de mettre en place un process de salle blanche permettant de caractériser les nanofils : les intégrer à une structure permettant des mesures. Pour cela, il faut comprendre les différents dispositifs de salle blanche. De plus, avant de pouvoir complètement caractériser les nanofils, il faudra réaliser de nombreux tests sur des échantillons différents afin de mettre en place le process flow à ce dont nous avons besoin puis des tests sur des nanofils pour l'ajuster.
