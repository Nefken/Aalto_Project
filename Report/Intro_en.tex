\begin{center}
\vspace*{\stretch{2}}
{\Large \textsc{Aknowledgments}}
\vspace{0.5cm}

\textit{First of all, I would like to thank Pr. Jukka Pekola for having accepted me within the PICO Group and having given me the possibility to do this internship. Of course this would not have been possible without Dr. Clemens Winckelmann who gave me Pr. Pekola contact details. Then, I will of course thank the whole PICO Group for having integrated me so well and helped me when I needed it, but especially Dr. Mathieu Taupin who I worked with on the nanowires project and who gave me precious advices about clean room processes and laboratory methods, alongside with Dr. Matthias Meschke on that last point. Finally, I would like to thank both \textsc{Grenoble INP Phelma} for having given me the opportunity to do such an internship and Aalto University which made it possible afterall.}
\vspace*{\stretch{9}}
\end{center}
\newpage

\section*{The PICO Group}

\indent The PICO Group is a research team which work under the responsibility of the Aalto University School of Science. Its facilities lies in the Micronova building, situated on the Aalto Campus of Otaniemi, in Espoo. Otaniemi is the centre of all scientific activites of the Aalto University and the Micronova building holds the major part of research about nano and microphysics in Finland. The team consists of nineteen members for the moment, including, of course the professor Jukka Pekola, who leads the group and doctor Matthias Meschke who is in charge of the laboratory. Then, the team is divided between five post-doc, seven PhD students and five summer students. 

The Group's researches are varied but especially lies on superconductors properties. The main topics are linked with quantum thermodynamics and single electron transport through nano and microscaled devices. Among the group there are theoretical physicists who focus on the theoretical aspect of research : statistics that rule heat transfert between two nanoscaled resistors\cite{statistics}, for example, whereas some other are focused on practical research : realization of NIS thermometers, electron counting through devices... Moreover, the Group work actively with the Centre for Metrology and Accreditation because one of their main goal is to redefine the ampere, the unit used to measure electrical current\cite{ampere}.

All theses topics of research can be reached thanks to the devices the Group and the building hold. The building owns a clean room, widely used by the members of the Group to make structures. The devices they use are mostly the Electron Beam Lithographier(EBL), evaporators LISA and MASA, and the Scanning Electron Microscope(SEM). The clean room also have devices for making semiconductors-based structures and the Atomic Layer Deposition device was invented here. Then, the Group have a dedicated room in which there are many machines and especially several dilution cryostats. Indeed, since most of the work done within the group has a link with superconductors, they need to reach low temperature to have access to this particular state of matter\cite{NIS_thermo}. There are three regular dilution cryostats that can reach temperature as low as 20mK, and a BlueFors cryostats that works differently but can reach even lower temperatures. In other rooms, there are also devices to characterize and measure : probestation, Atomic Force Microscope (AFM), and SEM, bounders... 

The University and these devices provide the Group the possibility to make research efficiently as we can see the numerous publications published every year in famous journals.


\chapter*{Introduction}
\addcontentsline{toc}{chapter}{Introduction}

Nanostructures are more and more complicated. Researcher in this field always tend to try to make sophisticated devices with many functionalities. This is possible thanks to clean room devices that provide the researchers a way to make more and more different processes. Yet, some of functionnalities in the devices are not well characterized since researchers tend to focus on their actual structures rather than on the characterization of the devices, especially when devices are quite new. However, it is important to have access to all the possibilities that a device offers, this is why it is important to characterize them. 

The evaporator LISA is quite a new device in Micronova's clean room, so that all its functionalities were not characterized, for example it allows plasma etching, which is an \textit{in situ} etching that can be use to get rid of a layer of matter that is not wanted on the structure. But, in order to make this etching reliable, its effect on samples need to be characterized : how does samples react to etching, does etching damage the samples, is the plasma uniform ? All these questions that need answers in order to be able to use the plasma properly.

My role among the group is to make and measure simple structures : Normal Metal-Insulator-Superconductor junction (NIS junction) to determine characterize the plasma etching in an evaporator.

First of all, I will need to understand the theory behind such structures since it is not part of my course, with a literature study (Chapter \ref{Chap1}).

Then, since I will fabricate some structures, I need to know how to do it. I will have trainings to use several devices in the clean room, but I need to understand how they really work, not just knowing on what button to push. Why do we use this device with this particular way to proceed, what are the parameters than have to be taken into account (Chapter \ref{Chap2}). 

Once I know how to the devices work, I need to set some parameters to make consistent measurements and characterize the other parameters with changing only one at a time : I need to build an experimental procedure. I also need to put in place a set up to make measurements : what type of measurements should be done, and how can we measure it (Chapter \ref{Chap3}).

Then, the measurements will give me some datas, which I will need to exploit to turn them into results which I can interpret to understand what happens (Chapter \ref{Chap4}).
