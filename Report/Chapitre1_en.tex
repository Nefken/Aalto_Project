\chapter{Theory}
 \label{Chap1}
 
    The first thing to do before starting anything is to understand the goal of the project. For that, it is important to study the behaviour of the structures we want to realize and the physics and quantum phenomemon that can occur. In this chapter we will sum up the litterature study that we have done to understand the theory behind the experiments.
        
        \section{Superconductivity}
            During the research project that Grenoble INP \textsc{Phelma} gave me the opportunity to realize during the first year of my master degree, I have studied superconductors with my group which make this study a bit easier for me since I am familiar with some of the concept mentionned here. Let's start with the basics. 
            Superconductivity is a state of matter which occur mostly at low temperature for several materials. It is a state where the material have an absolute zero resistance, so that current can run without energy losses. It is also a state where the material totally excludes magnetic field and becomes perfectly diamagnetic. These two major properties have a quantum explanation which is the Bardeen-Cooper-Schrieffer(BCS) Theory. Even of this theory does not make unanimity among physicist, since it does not explain everything, it is the most famous at the moment. The qualitative aspect of this theory is quite simple whereas the quantitative involves the second quantization and advanced quantum theories. The eletron within the matter can pair in so-called Cooper pairs which come from an interaction between eletrons an the ion lattice. At low temperature, electron are slow and they tend to attract ions. These ions have a relaxation time to come back to their initial state, but during the time they are in an non-equilibrium state, they create a local positive charge that can attract another electron. This electron is then paired with the previous one. It has the same implusion but an opposite spin according to the BCS Theory. The Cooper pairs, even if formed by two electrons are no longer fermions but bosons so that they follow the Bose-Einstein statistics are form a condensate within the matter. Then, the electrons that are not coupled follow a different density of states where there are two pikes at the distance of $\Delta$, around the energy of the condensate E$_0$ : This is the so-called superconductor gap. Theoretically, the pikes are infinite for a temperature of 0K.
            
        \section{NIS Junctions}
            
        
        \cite{NIS_thermo}
        
        \section{Dilution cryostat}
        
        
        
        %\section{Nanofils semiconducteurs}
        %La physique des nanofils est encore différente de celle des jonctions précédentes. En effet, le projet est basé sur l'utilisation de nanofils d'InAs, qui est semiconducteur. Cependant, celui-ci l'équipe de Copenhague le fait croître epitaxialement à de l'aluminium. Il subit ainsi une supraconductivité de proximité induite par celui-ci. C'est ce phénomène que l'on se propose d'étudier.
        
        %\section{Pompage électronique}
            %A partir des coulomb diamonds, haute fréquence, observation d'un courant d'un seul électron
            \begin{figure}
                \centering
                \begin{tikzpicture}[scale=1.2]
                \draw [->] (0,0)--(10,0);
                \draw (10,0) node[below] {$V_{g}$};
                \draw [->] (0,-3)--(0,3);
                \draw (0,3)node[left]{$V_{bias}$};
                \draw (0,1)--(1,2)--(2,1)--(3,2)--(4,1)--(5,2)--(6,1)--(7,2)--(8,1)--(9,2);
                \draw (0,-1)--(1,-2)--(2,-1)--(3,-2)--(4,-1)--(5,-2)--(6,-1)--(7,-2)--(8,-1)--(9,-2);
                \draw [dashed] (0,-1)--(2,1)--(4,-1)--(6,1)--(8,-1);
                \draw [dashed] (0,1)--(2,-1)--(4,1)--(6,-1)--(8,1);
                
                \draw (3,1)node{$n=0$};
                \draw (5,1)node{$n=1$};
                \draw (7,1)node{$n=2$};
                \draw (1,1)node{$n=-1$};
                \end{tikzpicture}
                
            \caption{Blocage de coulomb dans un supraconducteur et principe du pompage d'électrons}
            \end{figure}
            
